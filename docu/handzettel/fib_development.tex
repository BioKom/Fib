% Autor: Betti Oesterholz
% erstellt: 18.12.2010
% Dies ist der Handzettel zum Fib-Development Unternehmen
%
% Copyright (c) 2010 Betti Oesterholz
%
% Permission is granted to copy, distribute and/or modify this document
% under the terms of the GNU Free Documentation License, Version 1.2 or
% any later version published by the Free Software Foundation;
% with no Invariant Sections, no Front-Cover Texts, and no Back-Cover Texts.
%
% A copy of the license is included in the file ``fdl.tex'' .
%


\documentclass[12pt,a4paper]{article}
\pagestyle{empty} %keine Seitennummerierung
\usepackage{a4wide} % besser den Platz nutzen (kleinere R"ander)
\usepackage{ngerman} % Silbentrennung nach neuer Rechtschreibung
\usepackage[T1]{fontenc} % Type 1 Schriften
\usepackage{times} % Da die Standard-LaTeX-Schrift bei mir mit Acrobat nicht funkt.
\usepackage[ansinew]{inputenc} % Verwendung von Umlauten
%\usepackage{scrpage2} % Seitenformat
%\usepackage{titletoc}
%\usepackage{titlesec}
\usepackage{listings} % Programm-Listing
%\usepackage [usetoc]{titleref} 
\usepackage{graphicx} % Einbindung von Graphiken
\usepackage{url} % URLs durch \url{}
%\usepackage{bibgerm} % deutsche Bibliographie
%\usepackage{txfonts}%Paket fr das nat Symbol
\usepackage{multicol}
%\usepackage{makeidx}%for the index
%\usepackage{longtable}
%\usepackage{amsmath}%mathematikumgebung {equation*} usw.

% Seitenstil festlegen
%\pagestyle{useheadings}%or: headings

% naehste vier Zeilen: Format des Inhaltsverzeichnisses
%\titlespacing{\section}{0pt}{5cm}{5cm}%spacing by sections {in front}{above}{below}
%\dottedcontents{section}[1.5em]{\addvspace{1.0em}}{1.3em}{0.7pc}
%\dottedcontents{subsection}[3.8em]{}{2.2em}{0.7pc}
%\dottedcontents{subsubsection}[7.0em]{}{3.1em}{0.7pc}

%\setcounter{secnumdepth}{4}%%nummerierung der Unterabschnitte bis Tiefe


%%%%%%%%%%%%%%%%%%%%%%%%%%%%%%%%%%%%%%%%%%%%%%%%%%%%%%%%%%%%%%%%%%%%%%%%%%%%%%%%%%%%%%%
% Definitionen fr das Verwenden von Listings

\newtheorem{Def}{Definition}

%%%%%%%%%%%%%%%%%%%%%%%%%%%%%%%%%%%%%%%%%%%%%%%%%%%%%%%%%%%%%%%%%%%%%%%%%%%%%%%%%%%%%%%
% Beginn des Dokuments
%\makeindex

\begin{document}

%\renewcommand{\sectionmark}[1]{\markboth{#1}{}}
%\pagenumbering {Roman}
%\automark{section}
%\pagestyle{scrheadings} % individ. Seitenlayout
%\setheadsepline{0.4pt}
%\ihead{} % Titelzeile innen
%\ohead{} % Titelzeile aussen
%\chead{\slshape \headmark}  % Titelzeile mitte
%\cfoot{\thepage} % Fusszeile mitte

% braucht man ein Inhaltsverzeichnis, so sind die naehsten drei Zeilen auszukommentieren
%\setcounter{tocdepth}{3}
%\tableofcontents
%\clearpage

% Vorschlag fr Titelzeile:
% Bei umfangreicheren Dokumenten
%\ihead{\slshape \headmark } % Darstellung von Sectionnummer und -name
%\ohead{}
%\chead{}


%\pagenumbering{arabic}

\ \vspace{-2.5cm}
\begin{center}
	\LARGE\bf Fib-Development\\
\end{center}

\bigskip\noindent
Das Unternehmen Fib-Development dient als rechtlicher Rahmen zur F"orderung des Fib-Projekts. Es hat Institutscharakter, so dass nicht der Gewinn im Vordergrund steht, sondern das Fib-Multimediasystem.


\section{Angebot}

Das Fib-Multimediasystem er"offnet vielen Unternehmen und Organisationen, die Multimediadaten nutzen oder auf deren Nutzung aufbauen, neue Marktpotenziale. Damit diese Unternehmen die durch Fib entstehenden Vorteile besser nutzen k"onnen, ist die zentrale Koordinierung von Informationen, die Erforschung und die Entwicklung von Fib in einem Unternehmen (Fib-Development) sinnvoll.


\section{Ihr Vorteil}

"Uber den Zugang zu Informationen zu Fib k"onnen Unternehmen, die Fib verwenden, ihre Produkte verbessern und sich so Wettbewerbsvorteile verschaffen.

Die B"undelung der Entwicklung und der Verantwortung f"ur Fib in einem externen Unternehmen bringt f"ur Unternehmen folgende Vorteile:
\begin{itemize}
 \item Die Erfinderin und gr"o"ste Fachkompetenz Betti "Osterholz ist zentral an der Weiterentwicklung des Fib-Systems beteiligt.
 \item kompatibilit"at der auf Fib basierenden L"osung von unterschiedlicher Unternehmen
 \item Aufbau eines unabh"angigen Standards f"ur Fib
 \item geringere Ausgaben f"ur ein Unternehmen, da sich die Finanzierung "uber mehrere Unternehmen verteilen
 \item zugreifbarer Informationspool f"ur alle
 \item mehr Effizienz durch Spezialisierung
% \item Einzelnen Unternehmen kann nicht eine Einflussnahme auf dem Markt "uber Fib vorgeworfen werden. (Schutz vor Imagesch"aden)
% \item Einsparungen beim Personal und Aufwand, da nur ein Format f"ur mehrere Multimediainhalte zu ber"ucksichtigen ist.
\end{itemize}


\section{Unternehmensphilosophie: \newline F"ur Vielfalt. F"ur Wissen. F"ur Menschen!}

Grundziel des Unternhemens ist es, mit Hilfe von mehr Vielfalt, den Menschen ein besseres Leben zu erm"oglichen. Dazu ist die Schaffung von neuem Wissen n"otig, aus dem die neue Vielfalt erzeugt werden kann.

Fib ist ein System, das auf Vielfalt aufbaut. Um diese Vielfalt den Menschen zugute kommen zu lassen, werden vielf"altige Methoden genutzt, um die Entwicklung und Verbreitung des Fib-Systems voranzubringen.

\newpage
\section{Helfen Sie mit bei Fib}

Haupteinnahmequelle des Unternehmen sollte die Projektfinanzierung sein, wobei auch zus"atzliche produktbezogene Dienstleistungen gegen Entgelt angeboten werden.

\subsection{Investitionen}

Investitionen in Fib-Development werden der Weiterentwicklung und Verbreitung von Fib zugutekommen und damit jedem der Fib nutzt.

%Ungebundene "Uberweisungen, die dem Fib-System im Allgemeinen zukommen sollen, k"onnen direkt auf das Firmenkonto vorgenommen werden.
Gern k"onnen aber auch Zahlungen "uber Vertr"age geregelt werden, mit denen der Einsatz der Mittel festgelgt werden kann.

%\paragraph{Bankverbindung Firmenkonto:}
%
%\begin{tabbing}
%Kontoinhaberin: \hspace{1cm}\= Betti Susanne "Osterholz\\
%Kreditinstitut:\> Mittelbrandenburgische Sparkasse in Potsdam\\
%Bankleitzahl: \> 160 500 00\\
%Kontonummer: \> 10000 152 77\\
%\end{tabbing}
%\ \vspace{-1.0cm}

\subsection{Unterst"utzung}
%Hilfe: Informationen, Entwicklung (Rat und Tat), neue Operatoren, Kundenkontakte

Sie k"onnen aber auch die Weiterentwicklung und Verbreitung von Fib durch andere Mittel unterst"utzen.
Dies kann beispielsweise geschehen durch:
\begin{itemize}
 \item die Bekanntmachung des Fib-Systems
 \item die Vermittlung von Kunden/Investoren
 \item Verbesserungsvorschl"age zum Fib-System
 \item Entwicklungsleistungen
\end{itemize}
\ \vspace{-0.60cm}

\subsection{Fib-Development}

\begin{tabbing}
\hspace{35mm}\=\hspace{20mm} \=\kill
\textbf{Inhaberin:} \> Betti Susanne "Osterholz \>\\
 \> \>\\
\textbf{Adresse:}\> Falkenhorst 34 \>\\
	\>14478  Potsdam \>\\
	\>Tel.: \>0331 6012886\\
	\>E-Mail: \>\url{Oesterholz@fib-development.de}\\
	\>Webseite: \>\url{www.fib-development.de}\\
%	\>Private Webseite: \>\url{www.BioKom.info}\\
 \> \>\\
\textbf{USt-IdNr.:}\>DE 274116427 \>\\
\end{tabbing}




\end{document}


