% Autor: Betti Oesterholz
% erstellt: 09.2010
% Dies ist ein handzettel, welcher die Fib-Elemente beschreibt.
%
% Copyright (c) 2010 Betti Oesterholz
%
% Permission is granted to copy, distribute and/or modify this document
% under the terms of the GNU Free Documentation License, Version 1.2 or
% any later version published by the Free Software Foundation;
% with no Invariant Sections, no Front-Cover Texts, and no Back-Cover Texts.
%
% A copy of the license is included in the file ``fdl.tex'' .
%



\documentclass[10pt,a4paper]{article}
\pagestyle{empty} %keine Seitennummerierung
\usepackage{a4wide} % besser den Platz nutzen (kleinere R"ander)
\usepackage{ngerman} % Silbentrennung nach neuer Rechtschreibung
\usepackage[T1]{fontenc} % Type 1 Schriften
\usepackage{times} % Da die Standard-LaTeX-Schrift bei mir mit Acrobat nicht funkt.
\usepackage[ansinew]{inputenc} % Verwendung von Umlauten
%\usepackage{scrpage2} % Seitenformat
%\usepackage{titletoc}
%\usepackage{titlesec}
\usepackage{listings} % Programm-Listing
%\usepackage [usetoc]{titleref} 
\usepackage{graphicx} % Einbindung von Graphiken
\usepackage{url} % URLs durch \url{}
%\usepackage{bibgerm} % deutsche Bibliographie
%\usepackage{txfonts}%Paket fr das nat Symbol
\usepackage{multicol}
%\usepackage{makeidx}%for the index
%\usepackage{longtable}
%\usepackage{amsmath}%mathematikumgebung {equation*} usw.

% Seitenstil festlegen
%\pagestyle{useheadings}%or: headings

% naehste vier Zeilen: Format des Inhaltsverzeichnisses
%\titlespacing{\section}{0pt}{5cm}{5cm}%spacing by sections {in front}{above}{below}
%\dottedcontents{section}[1.5em]{\addvspace{1.0em}}{1.3em}{0.7pc}
%\dottedcontents{subsection}[3.8em]{}{2.2em}{0.7pc}
%\dottedcontents{subsubsection}[7.0em]{}{3.1em}{0.7pc}

%\setcounter{secnumdepth}{4}%%nummerierung der Unterabschnitte bis Tiefe


%%%%%%%%%%%%%%%%%%%%%%%%%%%%%%%%%%%%%%%%%%%%%%%%%%%%%%%%%%%%%%%%%%%%%%%%%%%%%%%%%%%%%%%
% Definitionen fr das Verwenden von Listings

\newtheorem{Def}{Definition}

%%%%%%%%%%%%%%%%%%%%%%%%%%%%%%%%%%%%%%%%%%%%%%%%%%%%%%%%%%%%%%%%%%%%%%%%%%%%%%%%%%%%%%%
% Beginn des Dokuments
%\makeindex

\begin{document}

%\renewcommand{\sectionmark}[1]{\markboth{#1}{}}
%\pagenumbering {Roman}
%\automark{section}
%\pagestyle{scrheadings} % individ. Seitenlayout
%\setheadsepline{0.4pt}
%\ihead{} % Titelzeile innen
%\ohead{} % Titelzeile aussen
%\chead{\slshape \headmark}  % Titelzeile mitte
%\cfoot{\thepage} % Fusszeile mitte

% braucht man ein Inhaltsverzeichnis, so sind die naechsten drei Zeilen auszukommentieren
%\setcounter{tocdepth}{3}
%\tableofcontents
%\clearpage

% Vorschlag fr Titelzeile:
% Bei umfangreicheren Dokumenten
%\ihead{\slshape \headmark } % Darstellung von Sectionnummer und -name
%\ohead{}
%\chead{}


%\pagenumbering{arabic}

\ \vspace{-2.5cm}
\begin{center}
	\LARGE\bf Fib-Elemente
\end{center}

\bigskip\noindent
\textbf{Vektoren:} Vektoren dienen zum Bereitstellen von numerischen Werten. Jedes Vektorelement ist entweder eine Zahl oder eine Variable, welche weiter oben im Fib-Objekt belegt wird. Zu den Vektoren geh"oren Definitionsbereiche, die im Allgemeinen im root-Element definiert werden.

\bigskip\noindent
\textbf{Variablen:} Variablen werden von einigen Fib-Elementen definiert und k"onnen dann im gesamten enthaltenen (bzw. darunter liegenden) Fib-Objekt verwendet werden.

\bigskip\noindent
\textbf{Fib-Objekte:} $Obj$

Fib-Objekte sind aus Fib-Elementen g"ultig zusammengesetzte Objekte.

\bigskip\noindent
\textbf{Punkte:} $Obj = p( Positionsvektor )$

Die Punkte sind die darstellenden Elemente. An ihnen werden die gesetzten Eigenschaften ausgewertet. Der leere Punkt hat keine Auswirkungen $p()$. Punkte mit leerem Positionsvektor $p(())$ erzeugen den Hintergrund.

\bigskip\noindent
\textbf{Eigenschaftselement:} $Obj = pr( Eigenschaftsvektor_{name}, Obj_1 )$

Mit dem Eigenschaftselement werden Eigenschaften (vom Type $name$) f"ur Fib-Objekte gesetzt.

\bigskip\noindent
\textbf{Listenelement:} $Obj = l( Obj_1, \ldots, Obj_n )$; $n \in N$ und $2 \geq n$

Mit dem Listenelement k"onnen mehrere Fib-Objekte zu einem Objekt zusammengef"ugt werden.

\bigskip\noindent
\textbf{Anmerkungselement:} $Obj = c( Schl"ussel , Wert , Obj_1)$

Das Anmerkungselement (comment) dient zum Benennen oder Beschreiben von Unterobjekten.

\bigskip\noindent
\textbf{Bereichselement:} $Obj = for( Variable, ( B_{1}, B_{2},\ldots, B_{n}), Obj_1)$

Das Bereichselement legt f"ur eine Variable die Werte aus dem Bereich der ganzen Zahlen (diskrete Bereiche) fest, welche sie einnimmt. Das Bereichselement enth"alt eine Liste von Unterbereichen $B_{i}$, welche die Variable annimmt.

\bigskip\noindent
\textbf{Funktionen:} $Obj = f( Variable ,UF ,Obj_1 )$

Funktionen sind Fib-Elemente, welche einer Variable einen Wert zuordnen, der mit Hilfe einer Formel berechnet wird.
Eine Funktion enth"alt daf"ur eine Unterfunktion $UF$.

\bigskip\noindent
\textbf{Externe Objekte aufrufen:}
\begin{eqnarray*}
Obj &=& obj( Identifierer , inVar_1 , \ldots , inVar_n , ( outVar_{1}, \ldots ,outVar_{v_1}, Obj_1), \ldots , \\
  && ( outVar_{1}, \ldots ,outVar_{v_m}, Obj_m) )
\end{eqnarray*}

Externe Objekte sind Fib-Objekte, welche nicht im aktuellen Fib-Teilobjekt definiert werden. Diese k"onnen aus dem Wurzelelement (root-Element) oder der Fib-Objekt\-daten\-bank kommen. Auf diese Weise k"onnen Teile von Fib-Objekten im gleichen Fib-Objekt mehrfach verwendet oder von verschiedenen Fib-Objekten wiederverwendet werden.

\bigskip\noindent
\textbf{Externe Unterobjekte:} $Obj = sub( Nummer , (var_1, \ldots , var_v) )$

Externe Unterobjekte sind Objekte, die beim Auswerten des aktuellen Fib-Objekts schon bereitgestellt werden (z. B. "uber externe Objekte).

\bigskip\noindent
\textbf{Root-Element:}
\begin{eqnarray*}
Rootobj &=& root( [Multimediainformation], [Domains], [DomainsValues], \\
&& [((inVar_1, S_1), \ldots , (inVar_v, S_v) )], [( outVarNum_{1}, \ldots , outVarNum_m)], Obj , \\
&& [((Identifier_1, Rootobj_1) , \ldots , (Identifier_n, Rootobj_n))],\\
&& [( DB\_Identifier_1, \ldots , DB\_Identifier_d)], [Optionalpart] )
\end{eqnarray*}

Das root-Element dient als Wurzelelement eines Fib-Objekts. Es sollte alle (Um\-ge\-bungs-)Informationen, welche f"ur die Auswertung des Fib-Objekts ben"otigt werden, bereitstellen. Das root-Element selbst kann in keinem Fib-Element, au"ser einem anderem root-Element, enthalten sein.


\end{document}







